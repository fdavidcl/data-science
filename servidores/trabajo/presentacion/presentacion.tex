%!TEX program = xelatex
% Importar configuración
\input{_preamble.tex}

\title{Introducción a la verificación formal}
\date{\today}
\author{David Charte}
\institute{Servidores seguros --- Universidad de Granada}

\begin{document}
\maketitle

\section{Introducción}
\subsection{Motivación}

\begin{frame}{Motivación}
\begin{columns}
\column{.5\textwidth}
\begin{itemize}
\item Requisitos de seguridad
\item Requisitos de temporalidad
\item Bugs!
\end{itemize}
\column{.5\textwidth}
\includegraphics[width=\textwidth]{bugs.png}
\end{columns}
\end{frame}

\begin{frame}{\contentsname}
  \setbeamerfont{section in toc}{size=\small}
  \setbeamerfont{subsection in toc}{size=\scriptsize}
  \vspace{1em}
  \tableofcontents
\end{frame}

\section{Proposiciones como tipos}
\subsection{Lógicas}

\begin{frame}{Lógica intuicionista}
Cada afirmación se debe probar de forma \textbf{constructiva}.

No asume \textbf{tercio excluso} ($p\vee \neg p$) ni eliminación de la doble negación $\Rightarrow$ No se puede demostrar por reducción al absurdo.

Utilidad: las definiciones partiendo de proposiciones bajo esta lógica son \textbf{computables}. No se puede definir
\begin{center}
\texttt{int n = if conjetura 1 else 0}
\end{center}
si no podemos probar la conjetura o su negación.

\end{frame}

\begin{frame}{Lógica de orden superior}

Lógica de \textbf{primer orden}: cuantificar elementos
\begin{center}
``existe $x$ tal que $x$ es mamífero y $x$ tiene pico''\\
$\bm{\exists x}(x\in M \wedge x\in P)$
\end{center}

Lógica de \textbf{segundo orden}: cuantificar relaciones entre elementos
\begin{center}
``existe una clase $O$ de animales tal que para cada animal $x$, si $x$ es mamífero y tiene pico entonces es de clase $O$''\\
$\bm{\exists O} \forall x (x\in M\wedge x\in P\rightarrow x\in O)$
\end{center}

Lógica de \textbf{orden superior} $=\bigcup\limits_n$ lógica de n orden.

Utilidad: mayor expresividad.

\end{frame}

\subsection{Teoría de tipos}

\begin{frame}{Teoría de tipos}
Estudia los sistemas formales que utilizan tipos para restringir las operaciones que se pueden aplicar a cada término.

Tipos de función: La función que lleva tipo \texttt{a} a tipo \texttt{b} es de tipo \texttt{(a -> b)}.

Hay una teoría de tipos intuicionista (\textbf{Martin-Löf}).

\end{frame}

\begin{frame}{Tipos dependientes}

Un tipo dependiente es un tipo cuya definición depende de un valor.

Ejemplo: ``plantilla de tipos'' \texttt{Matrix(m,n)}. Podemos definir la multiplicación de matrices
\begin{center}
\texttt{mult :: (Matrix(k,m), Matrix(m,n)) -> Matrix(k,n)}
\end{center}
o la función que devuelve un vector de $n$ elementos:
\begin{center}
\texttt{repNull :: n -> Vector(n)}\\
¡El tipo de retorno depende del valor del parámetro!
\end{center}

\end{frame}

\subsection{Sistemas de cálculo}

\begin{frame}{Sistemas de cálculo}

Son equivalentes (resuelven todo problema \textbf{efectivamente calculable}):
\begin{itemize}
\item Funciones recursivas generales (Gödel)
\item Cálculo lambda (Church)
\item Máquinas universales (Turing)
\end{itemize}

Es más sencillo razonar sobre estos sistemas que sobre un lenguaje de programación.

Cálculo lambda simplemente tipado: sólo incluye funciones y ``tipos básicos'', pero no es Turing-completo.
\end{frame}

%\begin{frame}{Turing-completitud}
%\end{frame}

\subsection{Correspondencia de Curry-Howard}

\begin{frame}{Correspondencia de Curry-Howard}
Cada tipo se relaciona con una proposición (Curry, 1934).

\begin{exampleblock}{Isomorfismo de Curry-Howard}
Una proposición verdadera se identifica con un tipo del cual existe al menos un objeto, y una proposición falsa corresponde a un tipo para el cual es imposible construir un objeto.
\end{exampleblock}

Si encontramos un objeto del tipo que buscamos, entonces tenemos una demostración para nuestra proposición!

\end{frame}

\begin{frame}{Correspondencia de Curry-Howard}

\begin{exampleblock}{Pareja / conjunción}
Sólo hay elementos de tipo \texttt{(a,b)} si los hay de \texttt a y de \texttt b. De igual forma, $a\wedge b$ sólo se demuestra con una demostración para $a$ y otra para $b$.
\end{exampleblock}

\pause

\begin{exampleblock}{Alternativa / disyunción}
Tipo \texttt{(a|b)} describe elementos que son de \texttt a o de \texttt b, se corresponde con $a\vee b$.
\end{exampleblock}

\pause

\begin{exampleblock}{Función / implicación}
\texttt{a -> b} indica entrada de un elemento de tipo \texttt a y salida de tipo \texttt b. Vistos $a$ y $b$ como proposiciones, existe tal función si y solo si se cumple $a\rightarrow b$.
\end{exampleblock}

\end{frame}

\begin{frame}{Correspondencia de Curry-Howard}
Ejemplos:
\begin{itemize}
\item Tautología: $a\rightarrow a$ es cierto así que siempre existe una función de tipo \texttt{a -> a} (la identidad).
\item Proyección: de una pareja podemos sacar el primer elemento (función tipo \texttt{(a,b) -> a}). Entonces, $a\wedge b\rightarrow a$ es verdadera.
\end{itemize}

\end{frame}

\section{Verificación formal}
\subsection{Enfoques}

\begin{frame}{Verificación de modelos}
Exploración \textbf{exhaustiva} de todos los
estados y transiciones del modelo matemático asociado al sistema.

Aplicable principalmente a sistemas con un \textbf{número finito de estados}, (también a algunos infinitos que se puedan representar finitamente). No adaptable a grandes sistemas.

Las propiedades verificadas mediante esta técnica suelen venir descritas en una \textbf{lógica temporal}:
``cada vez que ocurre x el sistema responde y''.
\end{frame}

\begin{frame}{Verificación deductiva}
Construir una \textbf{especificación formal} del
comportamiento de un sistema. Realizar demostraciones para deducir que la implementación la cumple.

Generalmente las demostraciones se resuelven
con
\begin{itemize}
\item \textbf{asistentes de demostración},
\item solucionadores de
teorías de satisfacibilidad módulo,
\item demostradores automáticos.
\end{itemize}
\end{frame}

\subsection{Asistentes de demostración}

\begin{frame}{Asistentes de demostración}
Programa que
ayuda al desarrollo de demostraciones formales mediante \textbf{colaboración entre el
humano y la máquina}.

Cada asistente incluye una teoría de tipos y generalmente utiliza \textbf{tácticas} para encontrar un elemento del tipo buscado (Curry-Howard: equivale a una demostración).

El humano
escribe una guía para la demostración, completada y verificada por la máquina.


\end{frame}

\begin{frame}{Asistentes de demostración}

\begin{itemize}
\item \textbf{Agda} (Haskell): lógica de orden superior, tipos dependientes
\item \textbf{Coq} (OCaml): lógica de orden superior, tipos dependientes
\item \textbf{Isabelle} (ML): lógica de orden superior, tipos simples
\item F*, HOL, PVS...
\end{itemize}


\end{frame}

\begin{frame}[fragile]{Ejemplo: Coq}
\scriptsize
\begin{verbatim}
(** Recursive polymorphic definition of trees *)
Inductive tree (X:Type) : Type :=
  | nilt : tree X
| node : X -> tree X -> tree X -> tree X.

(** Auxiliary lemmas about refl *)
Lemma refl_involutive : forall {X:Type} (t:tree X),
  refl (refl t) = t.
Proof.
  intros X t.
  induction t as [|a izq Hiz der Hdr].
    reflexivity.
    simpl. rewrite -> Hiz. rewrite -> Hdr. reflexivity.
Qed.
\end{verbatim}

Fuente: \url{https://github.com/M42/recorridosArboles}
\end{frame}

\section{Aplicaciones}

\subsection{Sistemas operativos}

\begin{frame}{Aplicaciones en sistemas}

Aplicable a kernels sencillos: derivados del \textbf{microkernel L4}.

Primeros intentos (1979): UCLA Secure Unix, Provably Secure Operating System. La verificación hacía el sistema \textbf{un orden de magnitud} más lento.

Actuales: PikeOS (real-time), VFiasco, seL4.

\end{frame}

\begin{frame}{Secure Embedded L4 (seL4)}

Primera demostración de corrección (200000 líneas para 7500 de código C) en 2009, sobre \textbf{Isabelle/HOL}.

Libre bajo GPLv2 en 2014.

Capas:
\begin{enumerate}
\item especificación abstracta (Isabelle)
\item especificación ejecutable (Haskell)
\item implementación (C) con prueba
\end{enumerate}

La línea de código de seL4 es \textbf{25x más barata} que la del promedio de programas que verifican \textit{Common Criteria}.

\end{frame}



\subsection{Software}

\begin{frame}{Aplicaciones en software}

Compiladores certificados: CompCert C (Coq)
{\scriptsize\url{https://github.com/coq/coq/wiki/List of Coq PL Projects}}

Criptografía verificada: HACL* (F*, parte de Project Everest)
{\scriptsize\url{https://blog.mozilla.org/security/2017/09/13/verified-cryptography-firefox-57/}}
\end{frame}

\subsection{Hardware}

\begin{frame}{Aplicaciones en hardware}

Síntesis de hardware: Fe-Si HDL (Coq)
{\scriptsize\url{https://link.springer.com/chapter/10.1007/978-3-642-39799-8_14}}

Propiedad de vivacidad en procesadores: Bus \textit{Runway} de HP (Isabelle)
{\scriptsize\url{https://rd.springer.com/chapter/10.1007/BFb0028385}}

Procesador DLX verificado: Verified Architecture Microprocessor (PVS)
{\scriptsize\url{https://link.springer.com/article/10.1007/s10009-006-0204-6}}

\end{frame}


\section{Conclusiones}
\subsection{Conclusiones}

\begin{frame}{Conclusiones}

La verificación formal presenta:
\begin{itemize}
\item Fuerte base teórica
\item Coste computacional alto
\item Mayor seguridad a menor coste económico
\item Enfoque más frecuente: deductiva con asistentes de demostración
\end{itemize}
\end{frame}

\begin{frame}[standout]
  ¡Gracias! ¿Preguntas?
\end{frame}
\end{document}
